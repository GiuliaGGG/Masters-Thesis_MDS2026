\documentclass{article}
\usepackage{graphicx} % Required for inserting images
\usepackage{xcolor}
\usepackage{float}

\usepackage[style=apa, backend=biber]{biblatex}


\addbibresource{references.bib}

\title{Master Thesis}
\author{Giulia Maria Petrilli}
\date{2025}

\begin{document}
\maketitle

\section{Introduction}

In late September 1880, close to a hundred farmers and peasants assembled at the estate of the local landlord Charles Cunningham Boycott in County Mayo, Ireland. They were protesting against the rent increases imposed by boycotts, and started to isolate the man and his family from business operations. 
The practice of organized non-consumption, now named boycott, has spread through the years, becoming a common tool for social movements to express their discontent towards companies or countries. The boycott, or threat of boycott,can have an effect when firms see a dent in their performance and adjust their behaviors to comply to the consumer's preference (\cite{Tomlin2019AssessingTE}; \cite{King2008APM}; \cite{McDonnell2013KeepingUA}). 
Groups have been organizing with this goal in mind in the context of the israeli occupation of Palestine.

The Boycott, Divestment, and Sanctions (BDS) is a nonviolent Palestinian-led movement promoting boycotts, divestments, and economic sanctions against companies that are perceived to be complicit in the Israeli occupation of the Palestinian territories (\cite{BDSWhatIsBDS}).
The movement aims to pressure these companies to change their policies and practices regarding the Israeli occupation of Palestinian territories.
BDS has been active since 2005, but with the start of the genocide in Gaza \footnote{as defined by Amnesty International \citeyear{Amnesty2024GenocideGaza}}, the movement has gained significant traction and visibility worldwide. As expression of that has been the surge in calls for economic boycotts \footnote{as proven by gained attention on social media platforms and google trends search term increasing}, a type of consumer activism where individuals or groups refuse to purchase products or services from specific companies to protest against their practices or policies.
One of the most notable boycotts in recent times has been against McDonald's, which has been accused of supporting the Israeli occupation through its business operations. More specifically, the initial trigger for the boycott was McDonald's Israel providing free meals to Israeli soldiers stationed in the occupied Palestinian territories (\cite{McDonaldsIL2023Meals}). Photos and videos of IDF (Israeli defence Force) soldiers with McDonald’s meals were widely circulated. Pro-Palestinian activists and social media users criticized the decision, interpreting it as an endorsement of the Israeli military’s operations in Gaza. In response, the hashtag \#BoycottMcDonalds rapidly gained traction across multiple countries, with particularly strong engagement in the Middle East and other Muslim-majority regions. Then, the BDS movement named McDonald’s as a boycott target, arguing that the actions of the Israeli franchisee constituted material support for the Israeli Defense Forces. On this basis, the company was framed as complicit and formally incorporated into the movement’s list of targeted boycotts.

Financial analysts and corporate disclosures indicate that boycott campaigns were associated with weaker-than-expected sales in certain regions, particularly in the Middle East (\cite{BHRRC_McDonalds_Boycott_2024} ; \cite{IrishTimes_McDonalds_Boycott_2025}). McDonald’s CEO Chris Kempczinski addressed the fallout during an earnings call in January 2024, directly linking sales declines to the boycott environment, reporting a meaningfully impacted performance (\cite{mcdonalds_q4_2023_earnings_call}). The current literature presents scarce examples of a robust empirical analysis of the impact of boycotts on company performance. Additionally, while the boycotts seem to have been more significant in International markets, the boycott calls operate in a digitalized and globalized setting, and this thesis inquires wether global sales were impacted, and not only the Middle eastern ones where people might feel closer to the cause because they are arab like palestinians. The vast open source presence of financial data from the U.S. Securities and Exchange Commission (SEC) online platform quarterly filings allows for a credible attempt to fill this gap. Attributing a decrease in company performanceas an effect of boycott calls risks to be an oversimplification, as other factors such as market competition, economic conditions, and changes in consumer behavior unrelated to boycotts could also play roles. This thesis investigates wether the boycott calls had consequences in the general revenue or McDonals, because BDS endorses the call for global boycotts, and because the movement has presence on social media, which has a global reach. Since the BDS movement largely operates through online platforms, this thesis will explore the interplay between digital activism and consumer behavior, examining whether online boycott campaigns influence purchasing decisions.

This thesis presents as application of the Synthetic Control Method (SCM) to formulate a credible counterfactual for McDonald's performance in the absence of a boycott. The paper's research question is: \textit{To what extent has the recognition of McDonald's as a boycott target by the BDS movement impacted the firm's global revenue?}. 

\paragraph{Null hypothesis ($H_0$).}
Boycotts have no causal effect on firm revenue. The revenue trajectories of boycotted firms do not differ from their counterfactual non-boycotted outcomes, neither immediately after the boycott call nor in subsequent periods.

\paragraph{Alternative hypothesis ($H_1$).}
Boycotts have a negative small causal effect on firm revenue. This effect may occur immediately following the boycott call and/or with a lag, after amplification of the global online boycott campaign.


\section{Literature Review}

To examine the existing research on the impact of economic boycotts calls on company performance is to delve into various disciplines. 
This thesis provides an account of the available literature on boycott calls, the BDS movement, and firm performance. 

\subsection{Boycotts and Boycott Calls}
The BDS movement was officially launched on 9 July 2005 when a coalition of more than 170 Palestinian civil society organizations, representing trade unions, refugee networks, women’s groups and other grassroots bodies, issued a “BDS Call” urging global boycotts, divestment and sanctions against Israel until it complies with international law and respects Palestinian rights. 
The political consumer boycott is the refusal to buy products from certain businesses in order to effect political or social change (\textcite{lee2012democratizing}).
Lee (2012) describes political consumer boycott as a well-suited tool of agency creation in a political landscape steered not by voting power, but by monetary power. Even if boycotts are now associated with the left-wing, the history of the phenomenon transcends political spectrum, contexts and historical realms. From Ghandi's Salt Satyagraha (the Salt march) to the Boston Tea Party, boycotts have been used as a leveraging tool for centuries. The CNN and the New York Times report a very recent example of Europeans boycotting U.S. products in response to the Trump administration, and  U.S. policies (\citeyear{cnn2025european_boycott} ; \citeyear{nyt2025us_boycott}).
Boycotts are long recognized as a threat to business by firm themselves, with McDonald's SEC annual data filings naming boycotts as a potential threats even from before the start of the current boycott calls from the BDS movement (\citeyear{McDonalds10K2007}). Tomlin et al. (\citeyear{Tomlin2019AssessingTE}) demonstrated that boycotts can have a statistically significant negative effect on shareholder wealth, with larger impacts for firms in competitive industries.
However, the evidence suggests the relationship is complex, with firm size, reputation, and market position also playing critical roles in determining the ultimate impact of boycott campaigns (\textcite{WangLiu2025_socialMediaBoycott}).

Perceived egregiousness, which is the extent to which the firm’s action is considered egregious, is the central trigger of boycott participation (Klein et al., \citeyear{klein2004corporate}). This reasearch interprets the start of the genocide in Gaza in october 2023 as a triggers of boycott participation, as it significantly increased the visibility and urgency of the BDS movement's calls for boycotts. 
Friedman's taxonomy of boycotts distinguishes between primary and secondary boycotts (\textcite{friedman2002consumer}). A primary boycott targets the party directly; a secondary boycott targets a secondary entity affiliated with the party (e.g., a supplier or distributor).
The BDS boycott calls against McDonald's can be classified as a primary boycott, as it directly targets the company directly.

The boycott call is the act of mobilizing consumers to participate in a boycott, often through social media campaigns, public demonstrations, and other forms of activism. Lasarov et al.(\citeyear{lasarov2023vanishing}) highlights that even when successful, consumer participation in a boycott decreases over time. In the context of the thesis, this might mean that even if the boycott calls were effective in the short term, their impact on McDonald's performance might show up as a spike in the immediate quarters after the boycott call, but then fade away as consumers return to their previous purchasing habits. it is interesting to observe that the boycott calls have shifted from being in person, or on newspapers, to being largely online, with social media playing a central role in mobilizing consumers.
Findings show that social media can amplify boycott impact, with Andrus and colleagues (\citeyear{Andrus2021_socialMediaBoycotts}) finding out that social media information directly influences stock market reactions. Further research shows that social movement organization formality, public demonstrations, and celebrity endorsements enable mediated disruption.(\textcite{King2011_tacticalDisruptiveness}).

\subsection{The Boycott, Divest, Sanction Movement}
The BDS movement was officially launched on 9 July 2005 when a coalition of more than 170 Palestinian civil society organizations, representing trade unions, refugee networks, women’s groups and other grassroots bodies, issued a “BDS Call” urging global boycotts, divestment and sanctions against Israel until it complies with international law and respects Palestinian rights. 

The literature shows interesting difference between the pre 2023 findings and the recent developments since the escalation, with the most recent literature showing a more positive outlook on the effectiveness of the movement.

The campaign has generated significant attention but achieved minimal tangible impact. D. Newman et al., 2016 notes that academic boycotts have been “almost insignificant” in disrupting scientific collaboration. I. Sheskin et al., 2016 suggests the global campaign has “had little success”, with Palestinian support declining from 59\% to 49\% between 2015 measurements. However, the movement is not without strategic value. Omar Barghouti et al., 2021 argues that Israel views BDS as a “strategic threat”, indicating potential political pressure. P. Di Stefano et al., 2014 draws parallels to successful South African anti-apartheid campaigns, suggesting nonviolent resistance can be effective.
Recent escalation of the genocide might have triggered boycott participation, in which case, it is worthwhile to explore the most recent developments of the movement in terms of effectiveness. The Crowd Counting Consortium reveals that from October 2023 to June 2024, nearly 12,400 pro-Palestine protests occurred in the United States, an unprecedented level of grassroots mobilization in support of Palestinian causes. (Deeb e Winegar, 2024). In the diaspora, mobilization efforts focused on advocating for policy changes in host countries, organizing BDS campaigns, and engaging in cultural and educational initiatives to raise awareness about the Palestinian cause (BDS Movement, 2023)
A more recent article Sopiyah et al., 2025 found that Instagram-based boycott campaigns succeeded in generating “economic impact on several brands included in the boycott list,” with high social media interaction encouraging public participation in boycotting Israel-affiliated products. The BDS movement has some ways of measuring its own effectiveness. 

\subsection{Firm performance}

The decision to consider a boycott successful or not ties into the broaderquestion of what success looks like for a movement. This is inevitably linked to what the ask is. 
The literature reveals that the choice of a dependent variable that measures boycott effectiveness is wide and varied. 
The first main choice is whether boycott impact is measured quantitatively or qualitatively. That is, some papers focus on measuring the actual economic impact of boycotts on firms, while others look at changes in corporate policies or practices as a result of boycott pressure.
Some other take negatives sentiments towards the company as a measure of effectiveness.
Within the realm on quantitatiev analysis, that this thesis places itself in, there is further heterogeneity in the choice of dependent variable. Share or stock prices are sometimes chosen to orperationalize company performance (\cite{pruitt1986determining}; \cite{eva2025impact}). While analyzing stock market price patterns for boycotted companies gives a good estimate on investor expectations about future firm value, it does not accurately capture consumer response. 
Kasaundra M. Tomlin et al., 2019 analyzed 125 U.S. boycotts from 1978 to 2017 and found CAR provided statistically significant evidence of boycott impact, with negative returns robustly confirmed through synthetic control methods. 
Revenue is also used as a measure of firm performance. Revenue captures the actual sales impact of boycotts, reflecting changes in consumer purchasing behavior. However, revenue can be influenced by various factors beyond boycotts, such as market trends and competition. The SEC definiton of revenue is ' Amount of revenue recognized from goods sold, services rendered, insurance premiums, or other activities that constitute an earning process. Includes, but is not limited to, investment and interest income before deduction of interest expense when recognized as a component of revenue, and sales and trading gain (loss)" \cite{sec_xbrl_revenues_63908_2025}. Hence, it is not a clean pictzre of revenue recognized from goods sold, and not a perfect indicator for boycott effect. 

in the context of this thesis, since the boycott trigger was an isolated event, the ask is unclear. The israeli mcdonalds franchise has received heavy backlash for their decision to give food to the IDF and they stopped provding suüport to the idk. mcdonalds also does not operate in the occupied territories, which is ofter a reason why brands are boycotted and the ask is often to stop doing that. Hence, the measure of what success looks like in th emcdonals case is up to speculation based on BDS posts and articles. Even if it was unproportional, one has to wonder why. Probably a statemebt distancing from the israel franchise. Probably the fact that mcdonalds has big power and chooses not to advance palestinian rights is something that enrages people. People are done with non partisianity, the fact that these companies try their very best to continue with business as usual amidst a genocide. 
The outcome variable for this paper is revenue. Amount of revenue recognized from goods sold, services rendered, insurance premiums, or other activities that constitute an earning process. Includes, but is not limited to, investment and interest income before deduction of interest expense when recognized as a component of revenue, and sales and trading gain (loss)."

\section{Data and Methodology}
The attempt to identify any sort of causation has one thinking counterfactually. In this sense, the question  \textit{To what extent has the recognition of McDonald's as a boycott target by the BDS movement impacted the firm's revenue?} asks what the revenue of McDonald's would have been in the absence of the boycott calls. A credible answer to these questions requires defendable choices in terms of the data and of the estimator used. 
The data used for both the control and the treated units comes from the SEC EDGAR database, the official U.S. Securities and Exchange Commission online platform where all public companies must file financial and regulatory documents.
Given that the platform is the primary disclosure system for US public companies, even if the quarterly data is unaudited, this thesis considers it reliable. The time frame used is quarterly data from Q1 2011 to Q3 2025. The data was scraped and preprocessed using Python, while the analysis was conducted using R, as the Synthetic Control Method has well established R packages. 
 
\subsection{Data Collection and Preprocessing}

Firm-level financial data were collected from the U.S. Securities and Exchange Commission (SEC) eXtensible Business Markup Language (XBRL) Company Concept APIs. XBRL is an XML-based format for reporting financial statements. The Company Concept APIs objects contain all disclosures related to the specified concept (e.g. revenue), organized by units of measure. Reported values correspond to XBRL facts associated with duration periods, defined by start and end dates, following the XBRL 2.1 specification (\cite{SEC_CompanyFacts_API}). Every reported value for that concept is organized in objects in the array, an example of which follows in figure 1. 

\begin{figure}[H]
    \centering
    \includegraphics[width=0.8\textwidth]{images/json_company_facts.jpeg}
    \caption{An object in SEC company concepts - Revenue }
    \label{fig:Figure 1}
\end{figure}

The way that the data is organized presents potential challenges for data users. These challenges can be grouped into aspects that simply make the data retrieval and processing tricky, and others that are structural of the data and may hinder the reliability of the analysis, if not properly accounted for.  

A structural problem with the data is that the start and the end date of the reporting periods are not aligned across quarters and across firms. SEC is transparent about this, and specifies that because company financial calendars can start and end on any month or day, and even change in length from quarter to quarter according to the day of the week, the data is assembled by the dates that best align with a calendar quarter or year (\cite{SEC_EDGAR_API}). This is a longstanding issue in the econometrics and finance field. A. Zellner et al., 1971 demonstrates that temporal aggregation can lead to multiple critical statistical limitations, including lower estimation precision, reduced test power, and an inability to capture short-run economic behaviors. N. Sinha et al., 2008 further reveals that firms’ reporting period choices are influenced by business seasonality and competitive information transfer effects, which can introduce additional variability. However, since this is a structural trait of financial filings, researcher still uses financial data despite discrepancies (\cite{Du2022LostIS} ; \cite{Chychyla2014UsingXT}). Additionally, Ryan T. Ball et al., 2018 demonstrate that aggregate earnings can be informative for macroeconomic forecasting, suggesting that minor reporting period misalignments do not necessarily invalidate the data’s usefulness. The quarterly data across firm reflect the same economic phase, and quarterly data are aggregate outcomes in the first place, so this thesis still uses them. To enable a comparison between quarterly units across firms, the data presents a unified quarterly time index that maps each firm's reporting dates to standard calendar quarters. Even when measures are taken, the fiscal heterogeneity of the SEC data entails measurement noise, and possibly a bias. Both have to be taken into consideration when interpreting the results of the analysis.  

As per the group of issues that make data retrieval tricky, researcher must me mindful about the following traits. The year and the quarter for each object, respectively fy and fp in figure 1, are reporting labels, not guarantees of temporal coverage. Hence, when interested in quarterly data, the most reliable measures of temporal coverage are start and end date. Figure 1 is an example of how an object might report values for 9 months and still be tagged as Q3, which may misleadingly suggest a temporal coverage of a quarter. Additionally, a context includes a period, which is defined as either a duration period (with a start date and an end date), or instant period (which only has an end date). Hence for the same fp and end date entries there might be several repetitions. The research had to take its own decision of how to reliably define a quarter, at the cost of being too conservative and cutting away relevant information. The SEC documentation specifies that quarterly data as a duration of 91 days, with a standard deviation of +/- 30 days (\cite{SEC_EDGAR_API}). This thesis consider a quarters an object that spans between 60 and 122 days, with the exception of Q4s. The retrieval of last quarter of the year presents further challenges, as often they are embedded in the annual reports, and it does not present a date that can show the reporting period. For the purpose of this analysis, the reseach opted to maintain only the quarters with a known start and end date, at the cost of losing some quarters embedded in yearly reports,.  

Additonally, firms often report economically equivalent concepts under different US-GAAP tags over time. This means, for example, that the variable that expresses revenue might be under different tags for some quarters or years, and this varies within and between firms alike. For this reason, raw XBRL tags were first mapped to a common set of semantic labels (e.g., revenue, net income, operating expenses). When multiple tags were available for the same firm and label, a dominant tag was selected based on reporting frequency, ensuring internal consistency of each firm’s time series. In some cases, the dominant tag existed for a defined time period, and was replaced by another one for another period (see figure below). In these less common cases, the research took the two main tags to ensure a smooth time trend for the variable. 

Having laid out the main issues and limitations in the data preprocessing, a practical description of how the data was scraped and preprocessed follows. This analysis' scraping file extracts the raw company concepts with all the tags available for the firms of interest. The preprocessing part loads the raw financial observations, then filter the quarterly intervals based on start and end date,  and proceeds with mapping the entries to fiscal quarter based on when the end date of the reporting period falls under. At this stage, the data presents several duplicates, because SEC/XBRL financial data typically contains multiple rows for the same underlying economic quantity because of different XBRL frames (contexts), amended filings (10-Q/A, 10-K/A), overlapping reporting windows, and other duplicate observations differing only in technical metadata. The script sometimes takes mechanical choiches, like chosing deterministically when the difference between rows is trivial, due to repetead filings of the same observation, for the purpose of maintaining one row per economic observation. For example, the column frame encodes reporting context, not economic content, and when the difference between rows in in the frame, only one observation is maintained. The script resolves the multiple reporting sources by selecting a dominant source tag, and goes on to preparing the data for the scm analysis. Treatment exposure was defined at the firm level using a binary indicator that switches on from the quarter in which the boycott event occurred. Then, the script refines the estimation window and drops chronically sparse donors. The is some missing data at this point, as it can be seen in figure below. The R packages tidysinth and asynth used further in the analysis do not tolerate unbalanced panels and missing entries, hence imputation is applied to every numeric column within each unit during the preprocessing by using forward/backward fill. Pictures of the pre and post imputation show how much of the data is imputed. 

The script standardizes firm-level revenue and related financial variables relative to each firm’s initial pre-sample level. For every firm, the first observed period is taken as a firm-specific baseline (period 0), and subsequent observations are expressed as ratios to this baseline value. This standardizazion is useful for capturing meaningful relative changes in revenue by removing large cross-firm differences in revenue scale, and supports the creation of a meaningful counterfactual. 

\subsection{Choiche of Estimator}

The study setup, pivotal for the estimator choiche, is as follows. The call to boycott mcDonalds sets a pre and port treatment period. There are a relatively small number of treated units (the quarterly equivalent of 2 years) and long pre-treatment histories (from 2011 to to 2023, so 12 years). There is a treated unit, McDonald, that receives a treatment that not every company receives, which is the call to boycott. Mcdonalds has several characteristics in common with other firms in the restauration and fast food business, ro that operate at a large global scale: being subjected to similar demands, fill a similar market, similar seasonality, economic shocks and inflation levels. Hence, they are somehow comparable. However, a comparison between McDonalds revenue and all the other firms' revenue would not be informative, because there are several other aspects that differentiate a fast food company from another, like different popularity, that make the revenues between different firms not comparable, or more specifically, not parallel before treatment. The treatment, a boycott call, is complex, endogenous, and non-linear. The setup and data used in this thesis lend themselves to a Synthetic Control Method (SCM), because mcdonalds has been the object of boycott calls and other companies have not, and simply checking the difference between a non boycotted company to McDonald after the treatment is not possible because there is no company with available data that matches well McDonalds pre treatment path and is not bocotted. The SCM is an extension of the simpler Difference in Difference (DiD) analysis, which compares the change in outcomes over time in a treated group to the change in outcomes over time in a control group to identify causal effects (\textcite{angrist2009mostly}), and is a good solution when assumption that the pre trend path of a control unit are parallel to the path of the treated and when there are significant time-varying unobserved confounders. Abadie et al (\citeyear{abadie2010synthetic}) developed the Synthetic Control Method (SCM) as an alternative to DiD when a combination of units provides a better comparison for the treated unit than any single unit alone. SCM a data-driven procedure that constructs a synthetic version of the treated unit by weighting a combination of control units (other fast-food companies not targeted by BDS). This synthetic control serves as a counterfactual, representing what would have happened to McDonald's revenue in the absence of the boycott. The SCM allows for the effects of unobserved variables on the outcome to vary with time, while DiD restricts them to be constant in time (\cite{abadie2010synthetic}). This A SCM is worthwhile exploring in this case also because there are many pre treatment units, 12 years worth of data. SCM does not require a large number of comparison units in the donor pool, which also fits this case as the post treatment is only of 2 years.

\subsubsection{Treatment Unit and Control Units}

The treated the firm, McDonald's, was chosen for several reasons. Firstly, the company has been a boycott target for around two consecutive years, offering several treated quarters. Secondly, McDonald's is among the boycotted companies with the most complete data availability on the SEC platform. Thirdly, several of the companies that are on the BDS boycott list have a primarly institutional performance, like tech companies that sell their units in bulk, for example. This makes it hard to capture consumer behavior through revenue changes. McDonald's, being a consumer-facing fast-food company, provides a clearer link between consumer activism and revenue impact. 

\begin{figure}[htbp]
    \centering
    \includegraphics[width=0.9\linewidth]{images/September_2025.png}
    \caption{Guide to BDS Boycott – September 2025}
    \label{fig:mcd_revenue}
\end{figure}

Since SCM does not use randomization, the treatment assignment is critical to ensure the validity of the results. The assignment process involves picking a poll of control units that could have been a boycott target, but were not. This thesis assumes that any company could be subject to boycotts eventually, because a statement supporting Israel or a partnership with the IDF are in the realm of possibility of virtually any company. It is also helpful to pick a pool of companies that could reasonably map out the treatment company to minimize unobserved confounders. The main two traits that McDonalds is essentialized with are that it is a fast food company, that it is costumers facing, and that it has a markedly global outreach. A typical SCM design identifies units that fit all the criteria when possible, and opts for a looser fit when data availability is limited. In the donor pool that this thesis chooses, some companyes are fast food companies, some have a global outreach, but no company has every trait that it needs to be a comfortable fit to Mcdonalds. This is far from ideal, but it is due to data availability constraints. In fact, many firms that would have been good control units either did not have publicly available data, or were boycotted in the first place, which I will elaborate more on in the discussion section. Find here a table with the donor pool, with a brief description of why the firm is considered a suitable donor.  

\begin{table}[H]
\centering
\caption{Donor Pool Composition and Rationale}
\label{tab:donor_pool}
\setlength{\tabcolsep}{4pt} % default is 6pt
\begin{tabular}{lll p{5cm}}
\hline
\textbf{Ticker} & \textbf{Company} & \textbf{Rationale for Inclusion} & \textbf{Limitations} \\
\hline
BBWI & Bath \& Body Works, Inc. & Consumer-facing retail brand & -- \\
BLMN & Bloomin' Brands, Inc. & Casual dining restaurant group & Only present in the US \\
CASY & Casey's General Stores, Inc. & Convenience and food-service retailer & Only present in the US \\
COST & Costco Wholesale Corporation & Large-scale consumer retailer & -- \\
DRI & Darden Restaurants, Inc. & Full-service restaurant operator & -- \\
GIS & General Mills, Inc. & Packaged food producer & -- \\
HD & The Home Depot, Inc. & Large consumer-facing retailer & Not a food firm; mainly present in the US \\
JACK & Jack in the Box Inc. & Quick-service restaurant chain & Mainly present in the US \\
LEVI & Levi Strauss \& Co. & Global consumer brand & Not a food firm \\
MAR & Marriott International, Inc. & Global hospitality firm & Not a food firm \\
MDLZ & Mondelez International, Inc. & Global branded snack food manufacturer & Not directly customer-facing \\
NFLX & Netflix, Inc. & Subscription-based consumer service & Not a food firm \\
TGT & Target Corporation & Mass-market consumer retailer & Not a food firm \\
TXRH & Texas Roadhouse, Inc. & Casual dining restaurant chain & Mainly present in the US \\
WEN & The Wendy’s Company & Quick-service restaurant peer & Mainly present in the US \\
\hline
\end{tabular}
\end{table}

\subsubsection{Pre-treatment Period and Post-treatment Period}

The Guide To BDS page (\citeyear{BDSGuide}), and variations thereof, have been informing users on what to boycott since the movement's inception. 
The pages shows images of brands that the movement decides to target once they fit their boycott criteria. Past presence on the page is verified through the Wayback Machine, an internet archiving service that captures and stores snapshots of web pages over time. The pre-treatment period is defined as the time before the BDS movement called for a boycott against McDonald's, which occurred in November 2023. A manual analysis of the wayback machine ensures that the boycott call was present since then, and not before, and also that the control group companies were not called for boycott at any point in time.

Mcdonalds is listed under BDS grassroots organic boycott targets. The organic boycott targets are campaigns that the BDS movement did not initiate, but is in support of them due to these brands being perceived as openly supporting Israel’s genocide against Palestinians (\cite{DS2025CompaniesGenocide}). The twitter post where mcdonalds israel gives food to the IDF is from the 19th of october 2023, and the BDS adding the result is on the 5th of November. This research acklowledges that BDS did not start the boycott, that the boycotts might have happened regardless of the BDS calls. However, taking the BDS call as a treatment is a way of centralizing this otherwise scattered treatment of a boycott call, this is a way that the research centralizes the treatment definition around a concrete and verifiable event. Additionally, the available literature says that is fair to assume that boycott movements need some sort of centralization to be effective. The next plot shows the treatment status of every firm. The x axist shows quarteras from 2011 to 2025, with the SCM readable entry 8ß96 corresponding to the last quarter of 2023. 

\begin{figure}[H]
    \centering
    \includegraphics[width=0.8\textwidth]{images/panelView_plot.png}
    \caption{Treatment Status}
    \label{}
\end{figure}

\section{Analysis}

The analysis presents an SCM model, an exploration of ASCM for bias correction in this setting and a DiD as a robustness check,to challenge the assumption of the initial model choice.

\subsection{SCM}
The analysis is conducted using the tidysynth package, which implements SCM but is preferred over the Synth package used in Abadie because it is more efficient and interpretable.  
The procedure constructs a synthetic counterfactual for MCD by selecting and weighting control firms such that their pre-treatment revenue trajectory closely matches that of the treated unit. The treatment is assumed to begin at 2023 Q3, and placebo units are generated to enable inference through permutation tests. The synthetic control is trained using only lagged values of the outcome variable (standardized revenue) over the pre-treatment period from 2011 Q1 to 2023 Q3. Optimal weights for the donor pool are chosen to minimize discrepancies in pre-treatment outcomes over this window, after which the synthetic control is constructed and used for post-treatment comparison. Unfortunately, the pre treatment fit between observed and synthetic path is very poor. 

\begin{figure}[H]
    \centering
    \includegraphics[width=0.8\textwidth]{images/sc_out_plot.png}
    \caption{Observed and Synthetic Revenue}
    \label{}
\end{figure}

The pre treatment units path is consistently well below the observed path, which indicates that, relative to the baseline period, McDonald’s experienced stronger revenue growth than the weighted combination of donor firms before treatment, suggesting either brand-specific growth dynamics or that McDonald’s was already on a more favorable growth trajectory compared to the donor pool, even after normalization. Additionally, the two paths and have an inverse trend, with observed ones going up when the synthetic goes down, which might indicate that that the synthetic control captures different cyclical or dynamics rather than firm-specific shocks affecting McDonald’s. This behaviour, however, might pont to the already raised issue that the start and the end date of the reporting periods are not aligned across quarters and across firms. This research maps them to common quarterly labels, however this might not have been enough to ensure that the quarters are comparable. 

There is a drop at the onset of covid, which means that the donor pool can capture aggregate shocks. However, this comes as no surprise, as close to any company experienced a drop during covid. Then, at around 2018, the trends swap, with the synthetic McDonalds being above the observed mMcDonalds. This might mean that McDonald’s underwent firm-specific changes around 2017–2018 that donors did not share, and this causes SCM to break even without treatment. This inversion in trends is a further issue in the identification strategy, because it means it means that there is a structural break in the relationship between McDonald’s and the donor pool, and that McDonald’s revenue trajectory diverged from the weighted average of donor firms in a way that the SCM could not reconcile using fixed pre-treatment weights. Plus, it implies the donor pool is not capturing McDonald’s exposure to the post-2018 times.

The distance between the two paths increases as time goes by. Since revenue is indexed to period 1, the y-axis reflects relative growth, not revenue levels, which implies that this  normalization amplifies divergence mechanically over long horizons. Small differences in growth rates may compound into large visual gaps.  

This thesis cannot argue that any divergence in the post treatment period are due to the boycotts. 

This plot shows the gap over time between observed McDonald’s revenue minus Synthetic McDonald’s revenue at each time period, clearly showing this 2018 inversion.

\begin{figure}[H]
    \centering
    \includegraphics[width=0.8\textwidth]{images/sc_plot_differences.png}
    \caption{Difference between the Synthetic and Observed Revenue for McDonald's}
    \label{}
\end{figure}

The next plot shows the the donor weights used to construct the synthetic McDonald’s. It shows which firms contribute to the synthetic control and by how much. Jack in the Box and Bloomin' Brands receive the highest weights, followed by Wendy's, indicating they are the most similar to McDonald's in terms of pre-treatment revenue patterns, and that the estimate hinges on these firm’s trajectory. The other brands weight close to 0. 

\begin{figure}[H]
    \centering
    \includegraphics[width=0.8\textwidth]{images/sc_plot_weights.png}
    \caption{Donor Weights for Synthetic McDonald's}
    \label{}
\end{figure}

The next plot shows McDonald’s treatment effect together with placebo effects from donor firms. Essentially, each donor is treated as if it were the treated unit. Since many placebos look similar, there is weak effect of any considerable effect.

\begin{figure}[H]
    \centering
    \includegraphics[width=0.8\textwidth]{images/sc_plot_placebos.png}
    \caption{Donor Weights for Synthetic McDonald's}
    \label{}
\end{figure}

\subsection{ASCM}
A way to solve a poor pre treatment fit is the ASCM (augmented SCM), designed for cases where SCM cannot achieve good pre-treatment fit. It “bias-corrects” imperfect matching by adding an outcome model, often ridge regression, that adjusts for remaining pre-treatment imbalance. ASCM might seem tempting because it effectively can reduce residual mismatch bias, which in SCM is very common as perfect pre treatment is hard to achieve. However, several aspects should warn a research off using this method, even if it may lead to statistically significant values. 
 
The Synthetic Control estimates indicate no statistically significant treatment effects across any post-treatment periods, with confidence intervals consistently including zero. The L2 imbalace is relatively high, suggesting that the donor-weighted synthetic control struggles to replicate McDonald’s revenue dynamics prior to treatment, and the average estimated treatment effect on the treated (ATT) is close to zero, which is coherent with the findings in the previous sections.
\begin{table}[H]
\centering
\caption{Synthetic Control (SCM) Estimates}
\label{tab:scm_results}
\begin{tabular}{lcccc}
\hline
\textbf{Time} & \textbf{Estimate} & \textbf{95\% CI (Lower)} & \textbf{95\% CI (Upper)} & \textbf{p-value} \\
\hline
8096 & -0.004 & -0.166 & 0.158 & 0.968 \\
8097 & -0.093 & -0.255 & 0.069 & 0.473 \\
8098 &  0.046 & -0.116 & 0.208 & 0.741 \\
8099 &  0.112 & -0.050 & 0.274 & 0.383 \\
8100 &  0.055 & -0.106 & 0.217 & 0.729 \\
8101 & -0.101 & -0.263 & 0.061 & 0.400 \\
8102 &  0.096 & -0.065 & 0.258 & 0.444 \\
\hline
\end{tabular}

\vspace{0.3cm}
\begin{tabular}{ll}
\textbf{Average ATT} & 0.0159 \\
\textbf{Joint Null p-value} & 0.81 \\
\textbf{L2 Imbalance} & 0.782 \\
\textbf{Improvement vs. Uniform Weights} & 91.1\% \\
\textbf{Inference Type} & Conformal inference \\
\end{tabular}
\end{table}

The Augmented Synthetic Control estimates yield larger post-treatment effects and improved pre-treatment balance relative to the canonical SCM, as reflected by a lower L2 imbalance and higher improvement over uniform weights. Several post-treatment periods exhibit statistically significant or marginally significant estimates, which may lead one to believe that the bias-corrected estimator identifies stronger divergence between observed and counterfactual outcomes once pre-treatment mismatch is adjusted for through a ridge regression outcome model.
However, these results rely on stronger modeling assumptions than the SCM. Given that divergence between observed and synthetic outcomes emerges prior to treatment, the estimated bias correction may be absorbing genuine structural differences rather than residual estimation error. Consequently, the ASCM findings are best viewed as sensitivity analyses illustrating how results change under model-based adjustments, rather than as definitive causal evidence. 

\begin{table}[H]
\centering
\caption{Augmented Synthetic Control (ASCM) Estimates}
\label{tab:ascm_results}
\begin{tabular}{lcccc}
\hline
\textbf{Time} & \textbf{Estimate} & \textbf{95\% CI (Lower)} & \textbf{95\% CI (Upper)} & \textbf{p-value} \\
\hline
8096 &  0.039 & -0.098 & 0.211 & 0.661 \\
8097 &  0.021 & -0.164 & 0.205 & 0.822 \\
8098 &  0.166 & -0.031 & 0.350 & 0.080 \\
8099 &  0.197 &  0.025 & 0.382 & 0.024 \\
8100 &  0.141 & -0.032 & 0.326 & 0.075 \\
8101 &  0.028 & -0.157 & 0.200 & 0.797 \\
8102 &  0.245 &  0.048 & 0.441 & 0.027 \\
\hline
\end{tabular}

\vspace{0.3cm}
\begin{tabular}{ll}
\textbf{Average ATT} & 0.119 \\
\textbf{Joint Null p-value} & 0.16 \\
\textbf{L2 Imbalance} & 0.501 \\
\textbf{Improvement vs. Uniform Weights} & 94.3\% \\
\textbf{Avg. Estimated Bias} & -0.104 \\
\textbf{Inference Type} & Conformal inference \\
\end{tabular}
\end{table}

ASCM does not solve a structural break problem, so the issues with the underlying mapping between McDonald’s and the donor pool. ASCM cannot  fix a bad donor pool problem, because if the donors are structurally different (some are from retail, subscription services, or hospitality), the model may end up fitting pre-periods but still extrapolate in a way that has no causal meaning. This comparison goes to show that even if a method like the ASCM migth seem tempting when pre treatment fit is poor, and might even yield statistically significant results like in this case, but it cannot fix structural issues with the data.

\subsubsection{DiD}

This section performs a DiD comparing McDonalds witht the revenue levels of the two companies that match it best, Jack in the boy and Wendy's. Since a SCM is unsuccessful, the purpose of this section is to challenge the initial assumptions of the DiD model, 

\section{Discussion}


I told you a lost of things, what ground conclusion do I come to, if any. why does it matter? why does this method work or not? 

Anyways, back to boycotts. What should the next person that writes about them do?
where should they go? 

What if the outcome variable wasn't revenue? 

What this research can takeaway from the data for boycotting purposes is that the sec one is difficoult to read if you're not a data scientist, so your average activist, which might be interest in this, can't understand it well. If you wanna get understandable one, there are plenty of services that sell it, and not for cheap. The third one is that the way that companies report their data might not be made to show boycott impact.

A fourth consideration ties to the difficoulty to find a proper donor pool, general thing that some companies are too big to fail, and they also end up being the ones that, proven by the fact that all the companies that could have provided a good donor, because similar in market and global outreach, were also boycotted. 

Since the data is aggregate, it might absorb smaller shocks like boycotts and make it impossible to see. While covid for example is pretty cler. 
DIRECT QUOTE, CHANGE THIS Because accounting lacks an all-embracing theoretical framework, dissimilarities in practices have evolved. As a consequence, net income is an aggregate of components which are not homogeneous. It is thus alleged to be a ``meaningless'' figure, not unlike the difference between twenty-seven tables and eight chairs. Under this view, net income can be defined only as the result of the application of a set of procedures $\{X_1, X_2, \ldots\}$ to a set of events $\{Y_1, Y_2, \ldots\}$ with no other definitive substantive meaning at all. Canning observes \cite{BallBrown1968}. 

McDonalds might be an unique company, whose synthetic control is difficoult to approximate with an aggregate of other companies. McDonalds is extremely capillar, mostly in the US but in the whole world, too. See McDonald's in Cuba on the Guantánamo Bay military base, or the controversial McCafe inside a former Taiwan leader's home in Hangzhou, China (\textcite{straitstimes_mcdonalds_taiwan_home} ; \textcite{mee_mcdonalds_interesting_locations}). 

The next biggest companies, which would have served also as a great control, like Domino's Pizza, Pizza Hut, Burger King, have also been subjected to boycotts. Several interesting firms, like Inspire Brands, which owns Dunkin’ among others, are privately held, and hence not on SEC data. 

Another challenge is closely related to the narture of corporate governance, which is the fact that some brands would have been interesting control group candidates, like, merge with other brands and get in multi-brand holding company whose data is only available in bulk. 

Given the data, a next thesis on this topic should focus on 

Further research should perform this analysys inthe international markets, because even if there is a strong correlation with gaza and the drop in sales there, it still would be worthwhile to investigate wether the synthetic control method performs well in a case where significance is easier to achieve, of if it would have been unsuitable in that case, too. 

\printbibliography

\section{Appendix}
\begin{table}[h!]
\centering
\begin{tabular}{p{3cm} p{11cm}}
\hline
\textbf{Field} & \textbf{Meaning} \\
\hline
\texttt{start} & Start date of the reporting period \\
\texttt{end}   & End date of the reporting period \\
\texttt{val}   & Reported numeric value for the concept \\
\texttt{accn}  & Accession number of the SEC filing where this value comes from \\
\texttt{fy}    & Fiscal year of the company \\
\texttt{fp}    & Fiscal period (\texttt{Q1}, \texttt{Q2}, \texttt{Q3}, \texttt{Q4}, or \texttt{FY}) \\
\texttt{form}  & SEC filing type (\texttt{10-Q}, \texttt{10-K}, etc.) \\
\texttt{filed} & Date the filing was submitted to the SEC \\
\hline
\end{tabular}
\caption{Description of fields in SEC financial reporting data}
\label{tab:sec_fields}
\end{table}


\end{document}
