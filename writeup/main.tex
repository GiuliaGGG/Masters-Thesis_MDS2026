\documentclass{article}
\usepackage{graphicx} % Required for inserting images
\usepackage{xcolor}
\usepackage[style=apa, backend=biber]{biblatex}

\addbibresource{references.bib}

\title{Master Thesis}
\author{Giulia Maria Petrilli}
\date{2025}

\begin{document}
\maketitle

\section{Introduction}

In late September 1880, close to a hundred farmers and peasants assembled at the estate of the local landlord Charles Cunningham Boycott in County Mayo, Ireland. They were protesting against the rent increases imposed by boycotts, and started to isolate the man and his family from business operations. 
The practice of organized non-consumption, now named 'boycott', has spread through the years, becoming a common tool for social movements to express their discontent towards companies or countries. The boycott, or threat of boycott,can have an effect when firms see a dent in their performance and adjust their behaviors to comply to the consumer's preference (\cite{Tomlin2019AssessingTE}; \cite{King2008APM}; \cite{McDonnell2013KeepingUA}). 
Groups have been organizing with thisgoal in ming in the context of the israeli oppucation of Palestine. 

The Boycott, Divestment, and Sanctions (BDS) is a nonviolent Palestinian-led movement promoting boycotts, divestments, and economic sanctions against companies that are perceived to be complicit in the Israeli occupation of Palestinian territories.
The movement aims to pressure these companies to change their policies and practices regarding the Israeli occupation of Palestinian territories.
BDS has been active since 2005, but with the start of the genocide in Gaza \footnote{as defined by Amnesty International \citeyear{Amnesty2024GenocideGaza}}, the movement has gained significant traction and visibility worldwide. As expression of that has been the surge in calls for economic boycotts \footnote{as proven by gained attention on social media platforms and google trends search term increasing}, a type of consumer activism where individuals or groups refuse to purchase products or services from specific companies to protest against their practices or policies.

One of the most notable boycotts in recent times has been against McDonald's, which has been accused of supporting the Israeli occupation through its business operations. More specifically, the initial trigger for the boycott was McDonald's Israel providing free meals to Israeli soldiers stationed in the occupied Palestinian territories (\cite{McDonaldsIL2023Meals}). Photos and videos of IDF (Israeli defence Force) soldiers with McDonald’s meals were widely circulated. Pro-Palestinian activists and social media users criticized the decision, interpreting it as an endorsement of the Israeli military’s operations in Gaza. In response, the hashtag \#\ BoycottMcDonalds rapidly gained traction across multiple countries, with particularly strong engagement in the Middle East and other Muslim-majority regions. Then, the Boycott, Divestment and Sanctions (BDS) movement subsequently named McDonald’s as a boycott target, arguing that the actions of the Israeli franchisee constituted material support for the Israeli Defense Forces. On this basis, the company was framed as complicit and formally incorporated into the movement’s list of targeted boycotts.

Financial analysts and corporate disclosures indicate that boycott campaigns were associated with weaker-than-expected sales in certain regions, particularly in the Middle East (\cite{BHRRC_McDonalds_Boycott_2024} ; \cite{IrishTimes_McDonalds_Boycott_2025}). McDonald’s CEO Chris Kempczinski addressed the fallout during an earnings call in January 2024, directly linking sales declines to the boycott environment: “The conflict in the Middle East has meaningfully impacted our business, particularly in some international markets. While we do not take sides in geopolitical conflicts, perceptions matter, and we are navigating through that.” While the message from the statement is clear, the current literature lacks a robust empirical analysis of the impact of boycotts on company performance. The vast open source presence of financial data from SEC quarterly filings allows for a credible attempt to fill this gap. Attributing a decrease in company performanceas an effect of boycott calls risks to be an oversimplification, as other factors such as market competition, economic conditions, and changes in consumer behavior unrelated to boycotts could also play roles. This thesis investigates wether the boycott calls had consequences in the general revenue or McDonals, because BDS endorses the call for global boycotts, and because the movement has presence on social media, which has a global reach. Since the BDS movement largely operates through online platforms, this thesis will explore the interplay between digital activism and consumer behavior, examining whether online boycott campaigns influence purchasing decisions.
This thesis presents as application of the Synthetic Control Method (SCM) to formulate a credible counterfactual for McDonald's performance in the absence of a boycott. The paper's research question is: \textit{To what extent has the recognition of McDonald's as a boycott target by the BDS movement impacted the firm's global revenue?}. 
The hypothesis is that boycotts have a small effect of revenue, possibly either right after the boycott is called, or with a lagged time effect for the global online campaign to gain outreach. 

\section{Literature Review}

To examine the existing research on the impact of economic boycotts calls on company performance is to delve into various disciplines. 
This thesis provides an account of the available literature on boycott calls, the BDS movement, and firm performance. 

\subsection{Boycotts and Boycott Calls}
The BDS movement was officially launched on 9 July 2005 when a coalition of more than 170 Palestinian civil society organizations, representing trade unions, refugee networks, women’s groups and other grassroots bodies, issued a “BDS Call” urging global boycotts, divestment and sanctions against Israel until it complies with international law and respects Palestinian rights. 
The political consumer boycott is the refusal to buy products from certain businesses in order to effect political or social change (\textcite{lee2012democratizing}).
Lee (2012) describes political consumer boycott as a well-suited tool of agency creation in a political landscape steered not by voting power, but by monetary power. Even if boycotts are now associated with the left-wing, the history of the phenomena transcends political spectrum, contexts and historical realms. From Ghandi's Salt Satyagraha to the Boston Tea Party, boycotts have been used as a leveraging tool for centuries. The CNN and the New York Times report a very recent example of Europeans boycotting U.S. products in response to U.S. policies (\citeyear{cnn2025european_boycott} ; \citeyear{nyt2025us_boycott}).
Boycotts are long recognized as a threat to business by firm themselves, with McDonald's SEC annual data filings nameing boycotts as a potential threats even from before the start of the current boycott calls from the BDS movement (\citeyear{McDonalds10K2007}). Tomlin et al. (\citeyear{Tomlin2019AssessingTE}) demonstrated that boycotts can have a statistically significant negative effect on shareholder wealth, with larger impacts for firms in competitive industries.
However, the evidence suggests the relationship is complex, with firm size, reputation, and market position also playing critical roles in determining the ultimate impact of boycott campaigns (\textcite{WangLiu2025_socialMediaBoycott}).

Perceived egregiousness, which is the extent to which the firm’s action is considered egregious, is the central trigger of boycott participation (Klein et al., \citeyear{klein2004corporate}). This reasearch interprets the start of the genocide in Gaza in october 2023 as a triggers of boycott participation, as it significantly increased the visibility and urgency of the BDS movement's calls for boycotts. 
Friedman's taxonomy of boycotts distinguishes between primary and secondary boycotts (\textcite{friedman2002consumer}). A primary boycott targets the party directly; a secondary boycott targets a secondary entity affiliated with the party (e.g., a supplier or distributor).
The BDS boycott calls against McDonald's can be classified as a primary boycott, as it directly targets the company directly.

The boycott call is the act of mobilizing consumers to participate in a boycott, often through social media campaigns, public demonstrations, and other forms of activism. Lasarov et al.(\citeyear{lasarov2023vanishing}) highlights that even when successful, consumer participation in a boycott decreases over time. In the context of the thesis, this might mean that even if the boycott calls were effective in the short term, their impact on McDonald's performance might show up as a spike in the immediate quarters after the boycott call, but then fade away as consumers return to their previous purchasing habits.
Findings show that social media can amplify boycott impact, with Andrus and colleagues (\citeyear{Andrus2021_socialMediaBoycotts}) finding out that social media information directly influences stock market reactions. Further research shows that social movement organization formality, public demonstrations, and celebrity endorsements enable mediated disruption.(\textcite{King2011_tacticalDisruptiveness}).

\subsection{The Boycott, Divest, Sanction Movement}
The BDS movement was officially launched on 9 July 2005 when a coalition of more than 170 Palestinian civil society organizations, representing trade unions, refugee networks, women’s groups and other grassroots bodies, issued a “BDS Call” urging global boycotts, divestment and sanctions against Israel until it complies with international law and respects Palestinian rights. 

The literature shows interesting difference between the pre 2023 findings and the recent developments since the escalation, with the most recent literature showing a more positive outlook on the effectiveness of the movement.


The campaign has generated significant attention but achieved minimal tangible impact. D. Newman et al., 2016 notes that academic boycotts have been “almost insignificant” in disrupting scientific collaboration. I. Sheskin et al., 2016 suggests the global campaign has “had little success”, with Palestinian support declining from 59\% to 49\% between 2015 measurements.However, the movement is not without strategic value. Omar Barghouti et al., 2021 argues that Israel views BDS as a “strategic threat”, indicating potential political pressure. P. Di Stefano et al., 2014 draws parallels to successful South African anti-apartheid campaigns, suggesting nonviolent resistance can be effective.
Recent escalation of the genocide might have triggered boycott participation, in which case, it is worthwhile to explore the most recent developments of the movement in terms of effectiveness. The Crowd Counting Consortium reveals that from October 2023 to June 2024, nearly 12,400 pro-Palestine protests occurred in the United States, an unprecedented level of grassroots mobilization in support of Palestinian causes. (Deeb e Winegar, 2024). In the diaspora, mobilization efforts focused on advocating for policy changes in host countries, organizing BDS campaigns, and engaging in cultural and educational initiatives to raise awareness about the Palestinian cause (BDS Movement, 2023)
A more recent article Sopiyah et al., 2025 found that Instagram-based boycott campaigns succeeded in generating “economic impact on several brands included in the boycott list,” with high social media interaction encouraging public participation in boycotting Israel-affiliated products. The BDS movement has some waysof measuring its own effectiveness. 

\subsection{Firm performance}

The decision to consider a boycott successful or not ties into the broaderquestion of what success looks like for a movement. This is inevitably sinked to what the ask is. 
The literature reveals that the choice of a dependent variable that measures boycott effectiveness is wide and varied. 
The first main choice is whether boycott impact is measured quantitatively or qualitatively. That is, some papers focus on measuring the actual economic impact of boycotts on firms, while others look at changes in corporate policies or practices as a result of boycott pressure.
Some other take negatives sentiments towards the company as a measure of effectiveness.
Within the realm on quantitatiev analysis, that this thesis places itself in, there is further heterogeneity in the choice of dependent variable. Share or stock prices are sometimes chosen to orperationalize company performance (\cite{pruitt1986determining}; \cite{eva2025impact}). While analyzing stock market price patterns for boycotted companies gives a good estimate on investor expectations about future firm value, it does not accurately capture consumer response. 
Kasaundra M. Tomlin et al., 2019 analyzed 125 U.S. boycotts from 1978 to 2017 and found CAR provided statistically significant evidence of boycott impact, with negative returns robustly confirmed through synthetic control methods. 
Revenue is also used as a measure of firm performance. Revenue captures the actual sales impact of boycotts, reflecting changes in consumer purchasing behavior. However, revenue can be influenced by various factors beyond boycotts, such as market trends and competition.

\section{Data and Methodology}
The attempt to identify any sort of causation has one thinking counterfactually. In this sense, the question  \textit{To what extent has the recognition of McDonald's as a boycott target by the BDS movement impacted the firm's revenue?} asks what the revenue of McDonald's would have been in the absence of the boycott calls. A credible answer to these questions requires defendable choices in terms of estimator, variable and treatment assignment.
The data used to contruct both the control and the treated units comes from the SEC EDGAR database, the official U.S. Securities and Exchange Commission online platform where all public companies must file financial and regulatory documents.
Given that the platform is the primary disclosure system for US public companies, even if the quarterly data is unaudited, this thesis considers it reliable. The time frame used is quarterly data from Q1 2008 to Q3 2025. The data was scraped and preprocessed using Python, while the analysis was conducted using R, as the Synthetic Control Method has a well established R packages. 
 
\subsection{Data Collection and Preprocessing}

Firm-level financial data were collected from the U.S. Securities and Exchange Commission (SEC) eXtensible Business Markup Language (XBRL) Company Concept APIs. XBRL is an XML-based format for reporting financial statements. The Company Concept APIs objects contain all disclosures related to the specified concept (i.e. Revenue), organized by units of measure. Reported values correspond to XBRL facts associated with duration periods, defined by start and end dates, following the XBRL 2.1 specification (\cite{SEC_CompanyFacts_API}). Every reported value for that concept is organized in objects in the array, an example of which follows in figure 1. 

\begin{figure}[htbp]
    \centering
    \includegraphics[width=0.8\textwidth]{images/json_company_facts.jpeg}
    \caption{}
    \label{fig:}
\end{figure}

The way that the data is organized presents potential challenges for data users. 

The first one is that the definition of what SEC considers a quarter to be is machine-readable but not researcher-readable by default. The year and the quarter for each object, respectively fy and fp in figure 1, are reporting labels, not guarantees of temporal coverage. Hence, when wanting to scrape quarters, the most reliable measures of temporal coverage are start and end date, which do not always correspond cleanly to fiscal quarters. Figure 1 is an example of how an object might report values for 9 months and still be tagged as Q3. Additionally, a context includes a period, which is defined as either a duration period, which has a start date and an end date, or instant period, which only has an end date, hence for the same value fp and end date there might be several value repetitions. The research had to take its own decision of how to reliably define a quarter, at the cost of being too conservative and cutting away relevant information. The SEC documentation specifies that quarterly data as a duration of 91 days, with a standard deviation of +/- 30 days (\cite{SEC_EDGAR_API}). This thesis consider a quarters an object that spans between 60 and 122 days, with the exception of Q4s. The retrieval of last quarter of the year presents further challenges, as often they are embeded in the annual reports, and it does not present a date that can show the reporting period. For the purpose of this analysis, the reseach opted to maintain only the quarters with a known start and end date, at the cost of losing some information.  

The second challenge when working with this data is that the start and the end date are not aligned across quarters and across firms. Because company financial calendars can start and end on any month or day, and even change in length from quarter to quarter to according to the day of the week, the data is assembled by the dates that best align with a calendar quarter or year (\cite{SEC_EDGAR_API}). Hence, start and end dates are not aligned, which creates issue when trying to align data across firms and create a balanced dataset for analysis. In this case, too, the research had to make its own decision on how to align data across firms, at the cost of losing some information. The solution was to create a unified quarterly time index that maps each firm's reporting dates to standard calendar quarters.

The third issue is that firms often report economically equivalent concepts under different US-GAAP tags over time. This means, for example, that the variable that expresses revenue might be under different tags for some quarters or years, and this varies within and between firms alike. For this reason, raw XBRL tags were first mapped to a common set of semantic labels (e.g., revenue, net income, operating expenses). When multiple tags were available for the same firm and label, a dominant tag was selected based on reporting frequency, ensuring internal consistency of each firm’s time series. In some cases, the dominant tag existed for a defined time period, and was replaced by another one for another period (see figure below). In these less common cases, the research took the two main tags to ensure a smooth time trend for the variable. 

\begin{figure}[htbp]
    \centering
    \includegraphics[width=0.8\textwidth]{images/nike_revenue_by_source_tag.png}
    \caption{}
    \label{fig: Example from a company }
\end{figure}

This analysis' scraping file extracts the raw company concepts with all the tags available for the firms of interest. The preprocessing part loads the raw financial observations, then filter the quarterly intervals based on start and end date. At this stage, the data presents several duplicates, because SEC/XBRL financial data typically contains multiple rows for the same underlying economic quantity because of different XBRL frames (contexts), amended filings (10-Q/A, 10-K/A), overlapping reporting windows, and other duplicate observations differing only in technical metadata. The scrip sometimes take mechanical choiches, like chosing deterministically when the difference between rows is trivial, for the purpose of maintaining one row per economic observation. For example, the column frame encodes reporting context, not economic content, and when the difference between rows in in frame, only one observation is maintained. The script resolves the multiple reporting sources by selecting a dominant source tag, and goes on to preparing the data for the scm analysis. Reporting dates were converted into a unified quarterly time index to ensure a unique and ordered temporal dimension across firms. Treatment exposure was defined at the firm level using a binary indicator that switches on from the quarter in which the boycott event occurred. Then, the script refines the estimation window and drops chronically sparse donors. The is some missing data at this point, as it can be seen in the panel Viewp pocture below. These are data that are timically imputed by the gsynth package used in the analyssys, but taking a long time, hence, the analysis chooses to impute numeric columns within each unit using forward/backward fill. 

The script standardizes firm-level revenue and related financial variables relative to each firm’s initial pre-sample level. For every firm, the first observed period is taken as a firm-specific baseline (period 0), and subsequent observations are expressed either as ratios to or deviations from this baseline value. By normalizing outcomes in this way, the transformation removes large cross-firm differences in revenue scale and focuses the analysis on relative changes over time. This is particularly relevant in the context of boycott analysis, where treatment effects are expected to manifest as proportional declines or recoveries in sales rather than as uniform level changes across firms. The resulting standardized variables capture boycott-related revenue dynamics as deviations from each firm’s own pre-boycott position, facilitating meaningful comparisons across firms of different sizes while avoiding reliance on linear relationships in revenue levels.

\begin{figure}[htbp]
    \centering
    \includegraphics[width=0.8\textwidth]{images/c83be82a-d34f-48dd-b5b6-0169ea44a1d4.png}
    \caption{Treatment Status}
    \label{fig:mcd_revenue}
\end{figure}

\begin{figure}[htbp]
    \centering
    \includegraphics[width=0.8\textwidth]{images/c83be82a-d34f-48dd-b5b6-0169ea44a1d4.png}
    \caption{Treatment Status}
    \label{fig:mcd_revenue}
\end{figure}

\begin{figure}[htbp]
    \centering
    \includegraphics[width=0.8\textwidth]{images/rawdata.png}
    \caption{Rawdata}
    \label{fig:mcd_revenue}
\end{figure}

\subsection{Choiche of Estimator}

The research question implies a time bound treatment andan outcome variable. This setup may allow a typical Difference in Difference (DiD) analysis, which compares the change in outcomes over time in a treated group to the change in outcomes over time in a control group,
identifying causal effects (\textcite{angrist2009mostly}). However, the setup and data used in this thesis lend themselves better to
a Synthetic Control Method (SCM), due to the violation of the parallel trends assumption present in the data, the amount of pre treatment units that the data offers, and the presence of time-varying unobserved confounders. SCM a data-driven procedure that constructs a synthetic version of the treated unit by weighting a combination of control units (other fast-food companies not targeted by BDS). This synthetic control serves as a counterfactual, representing what would have happened to McDonald's revenue in the absence of the boycott. 
the amount of pre treatment units that the data offers, and the presence of time-varying unobserved confounders.

DiD relies on the parallel trends assumption, which states that in the absence of treatment, the average change in the outcome variable would have been the same  for both the treated and control groups. In this case, the parallel trends assumption is not believable, as McDonald's pre-treatment trends differ from those of potential control units, due to it being a bigger company than most of the potential donors.
Abadie et al (\citeyear{abadie2010synthetic}) developed the Synthetic Control Method (SCM) as an alternative to DiD when the parallel trends assumption is violated, when  a combination of units provides a better comparison for the treated unit than any single unit alone. 

The data is taken from quarterly SEC fillings from Q1 2009 to Q3 2025, providing a total of 68 pre-treatment periods and 11 post-treatment periods. This extensive pre-treatment period allows for a construction of the synthetic control. SCM does not require a large number of comparison units in the donor pool.

A limitation of the DiD model is that, while it allows for the presence of unobserved confounders, it restricts their effect to be constant in time, so they can be eliminated by taking time differences (\cite{abadie2010synthetic}).The SCM extends the traditional linear panel data framework, allowing that the effects of unobserved variables on the outcome vary with time. This thesis uses the Synthetic Control Method (SCM) to estimate the impact of the BDS boycott calls on McDonald's revenue. Amount of revenue recognized from goods sold, services rendered, insurance premiums, or other activities that constitute an earning process. Includes, but is not limited to, investment and interest income before deduction of interest expense when recognized as a component of revenue, and sales and trading gain (loss)."

The analysis is conducted using the gsynth package, which implements the Generalized Synthetic Control Method (GSCM) developed by Xu (\citeyear{xu2017generalized}). GSCM extends the traditional SCM by allowing for time-varying unobserved confounders through a factor model structure. This is particularly useful in this context, as it is likely that there are unobserved factors affecting McDonald's revenue that vary over time, such as changes in consumer sentiment or broader economic conditions.


\subsection{Treatment Unit and Control Units}
The treated the firm, McDonald's, was chosen for several reasons. Firstly, the company has been a boycott target for around two consecutive years, offering enoguh treated quarters. Secondly, McDonald's is among the boycotted companies with the most complete data availability on the SEC platform. Thirdly, several of the companies that are on the BDS boycott list have a primarly institutional performance, like tech companies that sell their units in bulk, for example. This makes it hard to capture consumer behavior through revenue changes. McDonald's, being a consumer-facing fast-food company, provides a clearer link between consumer activism and revenue impact

\begin{figure}[h]
\centering

\begin{minipage}{\textwidth}
    \centering
    \includegraphics[width=0.9\linewidth]{images/January_2023.png}
    \\{\footnotesize (a) Guide to BDS Boycott - January 2023}
\end{minipage}

\vspace{0.6cm}

\begin{minipage}{\textwidth}
    \centering
    \includegraphics[width=0.9\linewidth]{images/September_2025.png}
    \\{\footnotesize (b) Guide to BDS Boycott – September 2025}
\end{minipage}

\caption{Comparison of the Guide to BDS Boycott across time.}
\label{fig:comparison}
\end{figure}

Since SCM does not use randomization, the treatment assignment, and the choice of the control group, is critical to ensure the validity of the results. In the context of this thesis, the assignment process involves picking a poll of control units that could have been a boycott target, but were not. 
The criteria to select treatment are based on the BDS page rationale for which companies become a boycott target, and are level of complicity, potential for cross-movement coalition-building, media appeal, and potential for success (\textcite{bdsStrategic2024}).
The research identifies companies that fit all the criteria when possible, and opts for a looser fit when data availability is limited.

\begin{table}[h!]
\centering
\begin{tabular}{l c c c c}
\hline
Company & Complicity & Coalition Building & Media Appeal & Potential \\
\hline
Walmart\\
Wendy's \\
Target \\
Johnson \& Johnson \\
Mondelez International\\ 
General Mills \\
Costco Wholesale \\
\hline
\end{tabular}
\caption{Control Units Selected for Synthetic Control Method}
\label{tab:basic}
\end{table}


\subsection{Pre-treatment Period and Post-treatment Period}
Since boycott calls are the treatment, and since the movement largely operates on their
social media and website, this thesis understands treatment as being present 
on the boycott brands page of the BDS website.

The Guide To BDS page (\citeyear{BDSGuide}), and variations thereof, have been informing users on what to boycott since the movement's inception. 
The pages shows images of brands that the movement decides to target once they fit their boycott criteria. This thesis assumes that once the brand appears on the website, social media action follows and user engagement on the page too. this is a way that the research centralizes the treatment definition around a concrete and verifiable event.
Past presence on the page is verified through the Wayback Machine, an internet archiving service that captures and stores snapshots of web pages over time.

The pre-treatment period is defined as the time before the BDS movement called for a boycott against McDonald's, which occurred in January 2023. the wayback machine was manually checked to ensure that the boycott call was present since then, and not before, and also that the control group companies were not called for boycott at any point in time. The post-treatment period starts from January 2023 and ends when the latest available data point, which is in Q3 2025.

\section{Preliminary Results}

\begin{figure}[htbp]
    \centering
    \includegraphics[width=0.8\textwidth]{images/c83be82a-d34f-48dd-b5b6-0169ea44a1d4.png}
    \caption{Treatment Status}
    \label{fig:mcd_revenue}
\end{figure}

\begin{figure}[htbp]
    \centering
    \includegraphics[width=0.8\textwidth]{images/plot.png}
    \caption{Treated vs Estimated Synthetic Control Revenue Over Time}
    \label{fig:mcd_revenue}
\end{figure}


\section{Discussion}


Further research should perform this analysys inthe international markets, because even if there is a strong correlation with gaza and the drop in sales there, it still would be worthwhile to investigate wether the synthetic control method performs well in a case where significance is easier to achieve, of if it would have been unsuitable in that case, too. 

\printbibliography

\section{Appendix}
\begin{table}[h!]
\centering
\begin{tabular}{p{3cm} p{11cm}}
\hline
\textbf{Field} & \textbf{Meaning} \\
\hline
\texttt{start} & Start date of the reporting period \\
\texttt{end}   & End date of the reporting period \\
\texttt{val}   & Reported numeric value for the concept \\
\texttt{accn}  & Accession number of the SEC filing where this value comes from \\
\texttt{fy}    & Fiscal year of the company \\
\texttt{fp}    & Fiscal period (\texttt{Q1}, \texttt{Q2}, \texttt{Q3}, \texttt{Q4}, or \texttt{FY}) \\
\texttt{form}  & SEC filing type (\texttt{10-Q}, \texttt{10-K}, etc.) \\
\texttt{filed} & Date the filing was submitted to the SEC \\
\hline
\end{tabular}
\caption{Description of fields in SEC financial reporting data}
\label{tab:sec_fields}
\end{table}

\begin{table}[h!]
\centering
\begin{tabular}{r r r}
\hline
\textbf{Firm} & \textbf{T0} & \textbf{T1} \\
\hline
1  & 68 & 0 \\
2  & 68 & 0 \\
3  & 68 & 0 \\
4  & 68 & 0 \\
McDonald's  & 57 & 11 \\
6  & 68 & 0 \\
7  & 68 & 0 \\
8  & 68 & 0 \\
9  & 68 & 0 \\
10 & 68 & 0 \\
11 & 68 & 0 \\
\hline
\end{tabular}
\caption{Treated and Control Units}
\end{table}


\end{document}
