\documentclass{article}
\usepackage{graphicx} % Required for inserting images
\usepackage{xcolor}
\usepackage[style=apa, backend=biber]{biblatex}

\addbibresource{references.bib}

\title{Master Thesis First Draft}
\author{Giulia Maria Petrilli}
\date{2025}

\begin{document}
\maketitle
\begin{abstract}

\end{abstract}


\section{Introduction}

In late September 1880, close to a hundred farmers and peasants assembled at the estate of the local landlord Charles Cunningham Boycott in County Mayo, Ireland. They were protesting against the rent increases imposed by boycotts, and started to isolate the man and his family from business operations. 
The practice of organized non-consumption, now named 'boycott', has spread through the years, becoming a common tool for social movements to express their discontent towards companies or countries. The boycott, or threat of boycott, has been proven to have an effect when firms see a dent in their performance and adjust their behaviors to comply to the consumer's preference, whose purchases depend on. 

The Boycott, Divestment, and Sanctions (BDS) is a nonviolent Palestinian-led movement promoting boycotts, divestments, and economic sanctions against Israel. 
It does so by promoting various forms of boycotts against companies that are perceived to be complicit in the Israeli occupation of Palestinian territories.
The movement aims to pressure these companies to change their policies and practices regarding the Israeli occupation of Palestinian territories.
BDS has been active since 2005, but with the start of the genocide in Gaza \footnote{as defined by Amnesty International \citeyear{Amnesty2024GenocideGaza}}, the movement has gained significant traction and visibility worldwide. As expression of that has been the surge in calls for economic boycotts \footnote{as proven by gained attention on social media platforms and google trends search term increasing}, a type of consumer activism where individuals or groups refuse to purchase products or services from specific companies to protest against their practices or policies.
One of the most notable boycotts in recent times has been against McDonald's, which has been accused of supporting the Israeli occupation through its business operations. 
Several articles and statements claim the effectiveness of boycotts on McDonald's, both by supporters of the movement and by financial analysts.
While a correlation exists, with several stores from boycotted brands closing and sales from said companies dropping as boycotts intensify, causation is delicate, and the current literature lacks a robust empirical analysis of the impact of boycotts on company performance. The vast open source presence of financial data from SEC quarterly filings allows for a credible attempt to fill this gap.
Studies show that pushing on social-political issues and communicating their stands in a way that speaks to their values could be rewarding for companies, 
earn a statistically significant stock return of 2.68\% in the four days immediately after their announcements (\textcite{afego2021does}).
This suggests that consumers are willing to reward companies that align with their values, and opens a venue for the opposite to happen, where consumers punish companies that do not align with their values.
This thesis places itself in the broader research corpus on consumer activism, which has gained momentum in recent years due to the growing awareness of social and environmental issues among consumers. The most notable case of boycott in recent history is the South African anti-apartheid movement, which used boycotts as a key strategy to pressure the South African government.
Differently, the current boycott called by BDS is bottom up, 
meaning that there is not a state ban on certain companies, 
but the phenomenon is rather grassroot. BDS mainly operates on social media. 
This makes a causal fit rather interesting, because wether in the south african
 case one can quite clearly connect th drop in sales to the ban,
  in this case wether one boycotts is fully subject fully to the consumer.
Delving in the potential causal effect of economic boycotts on company performance, operationalized as net income margin, sheds light on the effectiveness of consumer activism, and its ability to influence corporate behavior.


Perceived egregiousness, which is the extent to which the firm’s action is considered egregious, is the central trigger of boycott participation (Klein et al., \citeyear{klein2004corporate}). 
This reasearch interprets the start of the genocide in Gaza in october 2023 as a triggers of boycott participation, as it significantly increased the visibility and urgency of the BDS movement's calls for boycotts. 
Friedman's
taxonomy of boycotts distinguishes between primary and
secondary boycotts (Friedman,  \citeyear{friedman2002consumer}). A primary boycott targets the party directly; a secondary boycott targets a secondary entity
affiliated with the party (e.g., a supplier or distributor).

Boycott calls are 


The boycott calls success on social media might be a sign of virtue signaling, but not of actual impact on sales.

The BDS boycott calls against McDonald's can be classified as a primary boycott, as it directly targets the company directly.
This thesis provides a methodology to achieve such goal, using a Synthetic Control Method to formulate a credible counterfactual for McDonald's performance in the absence of a boycott. The paper's research question is: \textit{To what extent do the BDS boycott calls impact McDonald's VARIABLE?}. 

\section{Literature Review}
 
Studies to measure the effectiveness of economic boycott are not only on boycotts on Israel. This literature review examines the existing research on the impact of boycotts on firm outcomes.

\subsection{Boycott Calls and Firm Performance}

The political consumer boycott is the refusal to buy products from certain businesses in order to effect political or social change (\textcite{lee2012democratizing}).
Lee (2012) describes political consumer boycott as a well-suited tool of agency creation in a political landscape steered not by voting power, but by monetary power.
Even the McDonald's SEC data filings list boycott as a potential threat to the business.
The literature reveals that the choice of a dependent variable that measures boycott effectiveness is wide and varied. 
Some papers uses the change of a policy as a measure of effectiveness, while others focus on changes in sales or market share.
A paper has done this analysis but using share prices, which measures trust in a company and not if people buy it or not
The current literature on the impact of boycotts measures company performance using share or stock prices (\cite{pruitt1986determining}; \cite{eva2025impact}).
While analyzing stock market price patterns for boycotted companies gives a good estimate on the trust the investors place in them, it does not accurately capture consumer response. 

Lasarov et al.(\citeyear{lasarov2023vanishing}) highlights that even when successful, consumer participation in a boycott decreases over time. In the context of the thesis, this means that even if the boycott calls were effective in the short term, their impact on McDonald's performance might show up as a spike in the immediate quarters after the boycott call, but then fade away as consumers return to their previous purchasing habits.


The Boycott, Divest, Sanction (BDS) movement has been deploying their advocacy effort into companies that are targeted for boycotts.

the empirical evidence on boycotts and firm Outcomes
There are several challenges in identifying boycott effects
key methodological papers
this thesis contributes by

\section{Data and Methodology}
The data used to contruct both the control and the treated units comes from the SEC EDGAR database, the official U.S. Securities and Exchange Commission online platform where all public companies must file financial and regulatory documents.
Given that the platform is the primary disclosure system for US public companies, even if the quarterly data is unaudited, this thesis considers it reliable. 
The time frame used is quarterly data from \textcolor{red}{Q1 2009} to Q3 2025.
Gunning (\citeyear{edgartools}) provides tools for accessing SEC filings through a GitHub repository. The code provided was adapted to fit the needs of this analysis.


Outcome variable: \textcolor{red}{might need to provide my own indicator}
The independent variables are 

\subsection{Synthetic Control Method (SCM)}
The research question \textit{To what extent do the BDS boycott calls impact McDonald's VARIABLE?} implies a treatment,
an outcome variable, and a before and after periods. This setup may allow a typical Difference in Difference (DiD) analysis,
which compares the change in outcomes over time in a treated group to the change in outcomes over time in a control group,
identifying causal effects (\textcite{angrist2009mostly}). However, the setup and data used in this thesis lend themselves better to
a Synthetic Control Method (SCM), a data-driven procedure that constructs a synthetic version of the treated unit by weighting a 
combination of control units (other fast-food companies not targeted by BDS). This synthetic control serves as
 a counterfactual, representing what would have happened to McDonald's \textcolor{red}{net income margin} in the 
 absence of the boycott. This method is preferred to a simple Difference in Difference due to the violation of the parallel trends assumption present in the data, the amount of pre treatment units that the data offers, and the presence of time-varying unobserved confounders.
the amount of pre treatment units that the data offers, and the presence of time-varying unobserved confounders.

DiD relies on the parallel trends assumption, 
which states that in the absence of treatment, the average change in the outcome variable would have been the same 
for both the treated and control groups. In this case, the parallel trends assumption is not believable,
as McDonald's pre-treatment trends differ from those of potential control units, due to it being a bigger
company than most of the potential donors.
Abadie et al (\citeyear{abadie2010synthetic}) developed the Synthetic Control Method (SCM) as an
 alternative to DiD when the parallel trends assumption is violated, when 
a combination of units provides a
better comparison for the treated unit than
any single unit alone. 

The data is taken from quarterly SEC fillings from Q1 2009 to Q3 2025, providing a total of 68 pre-treatment periods and 11 post-treatment periods. 
This extensive pre-treatment period allows for a construction of the synthetic control. 
SCM does not require a large number of comparison units in the donor pool.


Another limitation of the DiD model is that, while it allows for
the presence of unobserved confounders, it restricts their effect to be constant in time, so they can be eliminated 
by taking time differences (\cite{abadie2010synthetic}).The SCM extends the traditional linear panel
dataframework, allowing that the
effects of unobserved variables on the outcome vary with time. 
This thesis uses the Synthetic Control Method (SCM) to estimate the impact of the BDS boycott calls on 
McDonald's \textcolor{red}{net income margin}. 
 
\subsection{Treatment Unit and Control Units}
The treated Unit has been trated for x amount of time, this table shows the

The control units, so the boycotted companies, have been exclusively Israel companies for whose public data his not freely accessible,
 and the movement has only recently pivoted from calling to boycotts more global brands, like Coca cola, Airbnb, and McDonald's, among others. 
While speaking of the evolution of the movement objective and corporate structure, 
this also provides a before and after moment where 
McDonald's was not boycotted and then it was.
tech companies that sell in bulk and whose performance is primarly institutional. The rest of the companies (like Aibnb) did not have many pre treatment units. Hence, the companies that could've been in the treatment group where around five. 
A salient characteristic of the SCM is its ability to account for unobserved confounders that vary over time, making it particularly suitable for this analysis. Since SCM does not use randomization, the assignment process is critical to ensure the validity of the results. In the context of this thesis, the assignment process involves picking a poll of control units that could have been a boycott target, but were not. 
the targeted companies and rationale (\textcite{bdsStrategic2024}). The main points to select treatment are level of complicity, potential for cross-movement coalition-building, media appeal, and potential for success. This thesis uses the BDS criteria to select both the target company and the control group.
The research identifies companies that fit all the criteria when possible, and opts for a looser fit when data availability is limited.

\begin{table}[h!]
\centering
\begin{tabular}{l c r r r}
\hline
Company & Complicity & Coalition Building & Media Appeal & Potential \\
\hline
Item A & 12 & 3.5 & 4.2 & 5.1\\
Item B & 7 & 1.2 & 2.3 & 3.4 \\
Item C & 19 & 8.4 & 7.6 & 6.5 \\
\hline
\end{tabular}
\caption{Control Units Selected for Synthetic Control Method}
\label{tab:basic}
\end{table}
\subsection{Pre-treatment Period and Post-treatment Period}
Since boycott calls are the treatment, and since the movement largely operates on their
social media and website, this thesis understands treatment as being present 
on the boycott brands page of the BDS website.

The Guide To BDS page (\citeyear{BDSGuide}), and variations thereof, have been informing users on what to boycott since the movement's inception. 
The pages shows images of brands that the movement decides to target once they fit their boycott criteria. This thesis assumes that once the brand appears on the website, social media action follows and user engagement on the page too. this is a way that the research centralizes the treatment definition around a concrete and verifiable event.
Past presence on the page is verified through the Wayback Machine, an internet archiving service that captures and stores snapshots of web pages over time.

The pre-treatment period is defined as the time before the BDS movement called for a boycott against McDonald's, which occurred in January 2023. the wayback machine was manually checked to ensure that the boycott call was present since then, and not before, and also that the control group companies were not called for boycott at any point in time.
The post-treatment period starts from January 2023 and ends when the latest available data point, which is in Q3 2025.

\section{Results}
\section{Discussion}

BDS boycott calls were restricted to israeli brands, and ith tim, they started calling out amerisan brands, which gives to think in terms of how globalization.
Boycotts are harder to enforce when the target is a massive corporation with a global presence, as consumers may find it challenging to avoid all products associated with the company. This might mean that as companies become more conglomerates and monopolies, consumer actions might be harder to enforce.

\printbibliography

\section{Appendix}

\begin{table}[h!]
\centering
\begin{tabular}{r r r}
\hline
\textbf{Firm} & \textbf{T0} & \textbf{T1} \\
\hline
1  & 68 & 0 \\
2  & 68 & 0 \\
3  & 68 & 0 \\
4  & 68 & 0 \\
McDonald's  & 57 & 11 \\
6  & 68 & 0 \\
7  & 68 & 0 \\
8  & 68 & 0 \\
9  & 68 & 0 \\
10 & 68 & 0 \\
11 & 68 & 0 \\
\hline
\end{tabular}
\caption{Treated and Control Units}
\end{table}


\end{document}
