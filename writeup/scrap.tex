Studies show that pushing on social-political issues and communicating their stands in a way that speaks to their values could be rewarding for companies, 
earn a statistically significant stock return of 2.68\% in the four days immediately after their announcements (\textcite{afego2021does}).
This suggests that consumers are willing to reward companies that align with their values, and opens a venue for the opposite to happen, where consumers punish companies that do not align with their values.
This thesis places itself in the broader research corpus on consumer activism, which has gained momentum in recent years due to the growing awareness of social and environmental issues among consumers. The most notable case of boycott in recent history is the South African anti-apartheid movement, which used boycotts as a key strategy to pressure the South African government.
Differently, the current boycott called by BDS is bottom up, 
meaning that there is not a state ban on certain companies, 
but the phenomenon is rather grassroot. BDS mainly operates on social media. 
This makes a causal fit rather interesting, because wether in the south african
 case one can quite clearly connect th drop in sales to the ban,
  in this case wether one boycotts is fully subject fully to the consumer.
Delving in the potential causal effect of economic boycotts on company performance, operationalized as net income margin, sheds light on the effectiveness of consumer activism, and its ability to influence corporate behavior.
The boycott calls success on social media might be a sign of virtue signaling, but not of actual impact on sales.


BDS boycott calls were restricted to israeli brands, and ith tim, they started calling out amerisan brands, which gives to think in terms of how globalization.
Boycotts are harder to enforce when the target is a massive corporation with a global presence, as consumers may find it challenging to avoid all products associated with the company. This might mean that as companies become more conglomerates and monopolies, consumer actions might be harder to enforce.